\section{Introduction}

One of the critical components of connectomics--the field concerned with reconstructing the wiring diagram of the brain at nanometer resolution--is automatic segmentation of electron microscopy (EM) images of the brain.
With recent advances in EM acquisition techniques, neuroscientists can now generate a terabyte of EM image data every hour~\cite{richard2016imaging}.
These images are typically anisotropic with $4$nm resolution in the $xy$ plane with $30-40$nm between slices. 
At this resolution, we can see all of the axon terminals and dendritic spines, and the synaptic connections between them.
Accurate reconstruction of every neuron paired with synapse detection will enable us create a true graphical representation of the brain and further our understanding of the underlying circuitry.
These segmentation, or label, volumes assign an integer label to every voxel where voxels have the same label only if they belong to the same neuron.

The first complete reconstruction of animal brain occurred in the 1980s with an ambitious manual effort of the C. elegans worm\TODO{CITE}. 
This species has \TODO{300} neurons and manual labeling required \TODO{13} years of intensive work.
More recent research focuses on Drosophila flies~\cite{jovanic2016competitive}\TODO{CITE JANELIA}, rodents~\cite{richard2016imaging,suissa2016automatic}, and even humans\TODO{CITE}. 
With an EM throughput of one terabyte per hour, neuroscientists can image a cubic millimeter of data (one petabyte) in six months. 
At this scale, the manual reconstruction efforts of the 80s are simply infeasible.
Thus, researchers have turned to the deep learning revolution to provide automatic solutions to the segmentation problem.

Automatic reconstruction techniques need to be fast, scalable, and accurate.
Ideally, image acquisition is the bottleneck of the connectomics pipeline which requires reconstruction efforts of a terabyte per hour~\cite{haehn2017scalable}.
We can achieve this throughput with parallelization among several GPUs with fast reconstruction techniques~\cite{funke2017deep,parag2017anisotropic}.
One of the issues with these automatic techniques is that they use only local context in decision making and are agnostic about the underlying biological system.
Thus, these methods make errors at scale and currently require human proofreading and other bulky error correction techniques~\cite{haehn2017guided,error_correction_using_CNN}. 

We propose a novel region merging framework that takes as input and oversegmentation of an EM volume.
Our method imposes both local and global biological constrains onto the output segmentation to more close match the underlying structure of neuronal processes.
Additionally, our method is independent of image resolution and acquisition parameters, enabling its application to isotropic and anisotropic data without retraining.
Images generated by electron microscopes often differ in appearance because of variations in staining techniques~\cite{briggman2012volume}.
By removing the dependence of our algorithm on the input images, we greatly reduce the need for additional costly ground truth data for each new stack of images.
