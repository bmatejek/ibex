\section{Running Times}

\subsection{System}

All performance experiments ran on an Intel Core i7-6800K CPU 3.40 GHz with a Titan X Pascal GPU. 
All code is written in Python and will be freely available upon the paper decision.
We use the Keras deep learning library for our neural networks with Theano backend and cuDNN 7 acceleration for CUDA 8.0. 

\subsection{Performance}

We train the network for $34$ epochs with each epoch taking approximately $1200$ seconds to complete.
Table \ref{table:performance} shows the total running time for each dataset for graph creation, CNN inference, and multicut partitioning. 
In all instances the time for these three steps combined is less than 100 seconds.
Currently extracting the skeletons with the TEASER algorithm is the bottleneck.
The skeleton for each label takes around $20$ to $30$ seconds to generate. 
However, the process is parallelizable as each segment can be processed independently.
Faster skeleton extraction is an important area for future research.
There is some existing research on fast skeleton generation and its applications to existing medical images (e.g. blood vessels, the colon)~\cite{palagyi2001sequential}.

\begin{table}[t]
	\centering
	\small
	\begin{tabular}{c c c } \hline
		& \textbf{Kasthuri Vol. 1} & \textbf{Kasthuri Vol. 2} \\ \hline
		Graph Creation & 34.76s & 35.90s \\
		Inference & 38.47s & 37.13s \\
		Multicut & 22.94s & 22.90s \\ \hline
		Total Time & 96.17s & 95.93s \\ \hline
		& \textbf{FlyEM Vol. 1} & \textbf{FlyEM Vol. 2} \\ \hline
		Graph Creation & 46.52s & 48.97s \\
		Inference & 12.55s & 10.91s \\
		Multicut & 30.30s & 30.14s \\ \hline
		Total Time & 89.37s & 90.02s \\ \hline
	\end{tabular}
	\caption{Performance of each step of the framework for all four datasets.}
	\label{table:performance}
\end{table}