\section{Conclusions}

We present a novel method for reconstructing neuronal processes in connectomics image datasets.
We extend on top of existing region-based agglomeration strategies and show significant accuracy improvement on two datasets from two different species. 
By extracting skeletons from every segment in the input we can quickly produce a graph $G$ with $N$ nodes and $E$ weighted edges from the input. 
We populate the edge weights using a 3-D CNN that, unlike previous methods, does not use the raw image data.
Separating the neural network from the images allows us to train on anisotropic data and test on isotropic data with different sample resolution. 
This is exceptionally important in connectomics because of the extreme labor-intensive cost of manually segmenting ground truth data. 
We use existing graph partitioning strategies on the $G$, enabling us to enforce domain-constraints on the resulting segmentation.
For us, our constraints follow from the topology of neurons.
However, this framework extends to other domains with varying constraints. 