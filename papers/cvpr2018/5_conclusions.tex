% !TEX root =  paper.tex
\section{Conclusions}

We present a novel method for reconstructing neuronal processes in connectomics datasets. We extend existing pixel-based reconstruction strategies using a morphological 3D skeleton graph and show significant accuracy improvement on two datasets from two different species. Our approach enables enforcing domain-constraints on the resulting segmentation. We believe this approach extends beyond connectomics to other domains with different constraints.

There is significant room for additional research following this topic. 
Finding the multicut partition of the graph provides some nice constraints on our reconstruction. 
However, we can further improve the results by tweaking the partitioning cost functions to more closely match the underlying biology.
For example, we can take the outputs of synaptic detectors to enforce that a given segment only has post- or pre-synaptic connections. 

Additionally, there is a lot of possible information we can extract from the skeletonization of the segments. 
Currently we use the skeletons to identify potential locations that should merge.
In the future, we can also find locations that are improperly merged in the input segmentation. 
Figure \ref{fig:negative-results} shows an error made by our algorithm that follows from a merge-error in the input. 
By training a CNN on the skeletons themselves, we can also receive information about whether a sequence of segments belong to the same neuron, rather than comparing pairs alone. 