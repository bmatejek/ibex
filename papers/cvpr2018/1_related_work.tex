\section{Related Work}

A significant amount of research in computer vision focuses on the segmentation of images \cite{zaitoun2015survey}. 
Here, we review some of the most successful methods that have been applied to large-scale EM images. Figure \ref{fig:pipeline} shows the results of a typical segmentation pipeline on these images. 

\paragraph{Pixel-based methods.} 

A large amount of connectomics research considers the problem of extracting segmentation information at the voxel level from the raw EM images to produce label volumes.
Some early techniques apply computationally expensive graph partitioning algorithms with a single node per voxel~\cite{andres2012globally}, however these methods do not scale to terabyte datasets.
Since then, several methods train 2-D CNNs to predict membrane probabilities per image slice~\cite{ciresan2012deep,jain2010boundary,kaynig2015large,seymour2016rhoananet,amelio_segmentation}. 
More recent strategies augment these neural networks with the $z$-dimension creating full 3-D CNNs~\cite{lee2015recursive,ronneberger2015u}.
Oftentimes these networks produce probabilities for the affinity between two voxels (i.e., the probability that adjacent voxels belong to the same neuron).
The MALIS cost function is specifically designed for generating affinities that produce good segmentations~\cite{briggman2009maximin}. 
More recently, flood-filling networks produce segmentations by training an end-to-end neural network that takes EM images to label volumes~\cite{januszewski2016flood}.
These flood-filling networks produce very impressive accuracies but at a high computational cost.


\paragraph{Region-based methods.} 

Several of these pixel-based approaches generate probabilities that neighboring voxels belong to the same neuron.
Many algorithms build on top of these methods and train random-forest classifiers to produce segmentations of the EM images~\cite{seymour2016rhoananet,nunez2014graph,10.1371/journal.pone.0125825,parag2017anisotropic,zlateski2015image}. 
Despite the success of these classifiers, they often make mistakes based on the local information leading to errors in the segmentation. 
Lately, many recent advancements in segmentation use artificial neural networks since machine-learned features outperform hand-designed ones~\cite{bogovic2013learned}.
A significant amount of research focuses on the training and optimization of these deep neural networks~\cite{chatfield2014return,maas2013rectifier,nesterov1983method}. 


\paragraph{Segment-based methods.} 

Some limited research builds on top of the region-based methods to correct errors in the segmentation \cite{rolnick2017morphological,error_correction_using_CNN,haehn2017guided}. 
However, to our knowledge, we are the first to extract a graph representation from these region-based methods. 
Many segmentation and clustering algorithms use graph partitioning techniques~\cite{andres2012globally} or normalized cuts for traditional image segmentation~\cite{kappes2016higher,shi2000normalized,tatiraju2008image}. 
We can enforce domain-specific constraints using graph-based paritioning algorithms.  
Unfortunately graph partitioning is an NP-Hard problem~\cite{demaine2006correlation}.
However, there are several useful multicut heuristics which provide good approximations with reasonable computational costs~\cite{horvnakova2017analysis,keuper2015efficient}.

