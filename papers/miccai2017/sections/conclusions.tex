\section{Conclusions} \label{sec:c}
% Things we learned during this project, such as:
% - LZMA is black magic
% - Certain encodings do not necessarily lead to better results after compression but can significantly improve other properties such as compression speed or before compression space consumption
% - Mitzenmacher is a cool guy
% - etc.
%In this paper, we study and evaluate encoding and compression methods for segmentation data with focus on connectomics.
%
%%% Summary
%In this paper, we study and evaluate compression of segmentation data for connectomics and introduce \appName---an efficient compression tool that outperforms existing solutions.
%Interestingly, we have learned that simple encoding schemes (e.g., RLE) can perform existing methods already and that complexity does not always proof beneficial. \cf{Since we have removed RLE from the plot we need to make sure to name it somewhere briefly in the results!}
%
%%% Future work
%Besides connectomics datasets, we show that \appName efficiently compresses labeled brain MRI data and we believe it generalizes well on any kind of labeled segmentation data.
%In the future we plan to improve random access to lower memory requirements for online viewers and focus on more efficient compression of the metadata (Sec. ~\ref{sec:m}).
%Also, we will integrate \appName into our analysis pipeline and provide wrappers for various other clients.
%
%%% Promotion
%To encourage testing of our tool, replication of our experiments, and adoption in the community, we provide \appName and results as free and open research at (link omitted for review).%\appUrl.
We have introduced \appName, an efficient compression tool for segmentation data that outperforms existing solutions on connectomics, MRI, and other segmentation data. 
%We believe our approach will generalize well to other kinds of label data. 
In the future we plan to improve random access to lower memory requirements for online viewers and enhance compression of the metadata.
Also, we will integrate \appName into our analysis pipeline and various end-user applications. 
To encourage testing of our tool, replication of our experiments, and adoption in the community, we release \appName and our results as free and open research at \href{github.com/rhoana/Compresso}{github.com/rhoana/Compresso}. 
%{\color{White} B to the Ockwurst!}

\smallskip

{\small M. Mitzenmacher is supported in part by NSF grants CNS-1228598, CCF-1320231, CCF-1535795, and CCF-1563710. H. Pfister is supported in part by NSF grants IIS-1447344 and IIS-1607800, by the Intelligence Advanced Research Projects Activity (IARPA) via Department of Interior/Interior Business Center (DoI/IBC) contract number D16PC00002, and by the King Abdullah University of Science and Technology (KAUST) under Award No. OSR-2015-CCF-2533-01.}
