% last updated in April 2002 by Antje Endemann
% Based on CVPR 07 and LNCS, with modifications by DAF, AZ and elle, 2008 and AA, 2010, and CC, 2011; TT, 2014; AAS, 2016

\documentclass[runningheads]{llncs}
\usepackage{graphicx}
\usepackage{amsmath,amssymb} % define this before the line numbering.
\usepackage{ruler}
\usepackage{color}
\usepackage{subfig}
\usepackage{times}
\usepackage{epsfig}
\usepackage{graphicx}
\usepackage{amsmath}
\usepackage{amssymb}
\usepackage{caption}
%\usepackage{subcaption}
\usepackage{booktabs}
\usepackage{makecell}
\usepackage{algorithm}
\usepackage{algpseudocode}
\usepackage{siunitx}
\usepackage{float}
\usepackage[width=122mm,left=12mm,paperwidth=146mm,height=193mm,top=12mm,paperheight=217mm]{geometry}




\begin{document}
% \renewcommand\thelinenumber{\color[rgb]{0.2,0.5,0.8}\normalfont\sffamily\scriptsize\arabic{linenumber}\color[rgb]{0,0,0}}
% \renewcommand\makeLineNumber {\hss\thelinenumber\ \hspace{6mm} \rlap{\hskip\textwidth\ \hspace{6.5mm}\thelinenumber}}
% \linenumbers
\pagestyle{headings}
\mainmatter
\def\ECCV18SubNumber{***}  % Insert your submission number here

\titlerunning{ECCV-18 submission ID \ECCV18SubNumber}

\authorrunning{ECCV-18 submission ID \ECCV18SubNumber}

\title{Graph-Based Neural Reconstruction from Skeletonized 3D Networks}

\author{Anonymous ECCV submission}
\institute{Paper ID \ECCV18SubNumber}

\newcommand\TODO[1]{\textcolor{red}{#1}}


\maketitle

\begin{abstract}
Advancements in electron microscopy image acquisition have created massive connectomics datasets in the terabyte range that make manual reconstruction of neuronal structures infeasible.
Current state-of-the-art automatic methods segment neural membranes at the pixel level followed by agglomeration methods to create full neuron reconstructions.
However, these approaches widely neglect global geometric properties that are inherent in the graph structure of neural wiring diagrams.
In this work, we follow bottom-up pixel-based reconstruction by a top-down graph-based method to more accurately approximate neural pathways.
We first generate skeletons in 3D from the neuron labels of the pixel-based segmentation.
We then simplify this skeletonized 3D network into a 3D graph with nodes corresponding to labels from the segmentation and edges identifying potential locations of segmentation errors.
We use a CNN classifier trained on ground truth data to generate edge weights on the 3D graph corresponding to error probabilities.
We then apply a multicut algorithm to generate a partition on the graph that improves the final segmentation.
Because the 3D graph is small and encodes top-down information our method is efficient and globally improves the neural reconstruction.
We demonstrate the performance of our approach on multiple real-world connectomics datasets with an average split variation of information improvement of 19.3\%.
\keywords{connectomics, skeletonization, deep learning}
\end{abstract}

\section{Introduction}

\begin{figure}
	\centering
	\includegraphics[width=0.42\linewidth]{./figures/intro-slice.png}
	\hspace{0.085\linewidth}
	\includegraphics[width=0.42\linewidth]{./figures/intro-cube.png}
\end{figure}

The field of connectomics concerns the wiring diagram of the brain at nanometer resolutions. 
Recent advancements in image acquisition using serial-section electron microscopy (ssEM) has allowed neuroscientists to produce a terabyte of electron micrscopy (EM) image data every hour~\cite{hildebrand2017whole}.
It is not feasible for domain experts to manually segment the vast amount of 3-D image data to model an entire brain.
Neuroscientists believe that reconstructing an entire mammilian brain at fine resolution will enable new insights into the workings of the brain~\cite{kasthuri2015saturated}. 
These observations will allow for new advancements in neuromedicine and artificial intelligence (CITE). 
Segmentation is part of this reconstruction process where we assign a unique label for every neuron in the EM image.

A significant amount of research focuses on automatic reconstruction of the neurons in these EM images because of the scope and importance of the problem. 
Several of these algorithms attempt to extract complete neurons through the 3-D volumes using only the raw image data as input.
Oftentimes, convolutional neural networks predict membrane probabilities or affinities between voxels and apply simple thresholds to agglomerate the voxels into clusters~\cite{lee2015recursive,ronneberger2015u}.
These \textit{per-voxel} algorithms produce useful outputs but currently fall short of complete reconstructions with sufficient accuracy. 

Researches currently address the failures of the per-voxel algorithms by training random-forest classifiers to agglomerate an oversegmentation of voxels~\cite{nunez2014graph} (CITE NEUROPROOF). 
These classifiers take the output of the per-voxel algorithms as input and generate high-level statistics such as affinity distributions between pixel regions. 
Presently, these methods use hand-designed features despite the evidence that machine-learned features perform better~\cite{bogovic2013learned}. 
These \textit{per-supervoxel} algorithms outperform the per-voxel algorithms but still do not provide the accuracy needed for large scale reconstruction of the brain.
Additionally, these methods do not fully leverage the wealth of shape information available. 

Here we present a \textit{shape-based} strategy that builds on top of the outputs of per-supervoxel methods. 
Similarly we take as input the output of the per-supervoxel methods. 
However, we focus on the overarching shapes traversing through the label volumes to identify potential merge candidates. 
From these locations we extract machine-learned features and generate a probability that two segments should merge. 
Lastly, we construct a graphical representation of the input label volume and apply a global segmentation algorithm to produce a better reconstruction.
Using graph-based optimization strategies enables us to enforce global constraints that more closely match the underlying biology of the images.

% !TEX root =  paper.tex
\section{Related Work}

A significant amount of research in computer vision focuses on the segmentation of images~\cite{zaitoun2015survey}. Here, we review some of the most successful methods that have been applied to large-scale EM images in connectomics. Fig.~\ref{fig:pipeline} shows the results of a typical connectomics segmentation pipeline.

\paragraph{Pixel-based methods.}
A large amount of connectomics research considers the problem of extracting segmentation information at the pixel (i.e., voxel) level from the raw EM images. Some early techniques apply computationally expensive graph partitioning algorithms with a single node per pixel~\cite{andres2012globally}. However these methods do not scale to terabyte datasets. More recent methods train classifiers to predict membrane probabilities per image slice either using 2D~\cite{ciresan2012deep,jain2010boundary,kaynig2015large,seymour2016rhoananet,amelio_segmentation} or 3D CNNs~\cite{lee2015recursive,ronneberger2015u,turaga2010convolutional}.

Oftentimes these networks produce probabilities for the affinity between two voxels (i.e., the probability that adjacent voxels belong to the same neuron). The MALIS cost function is specifically designed for generating affinities that produce good segmentations~\cite{briggman2009maximin}. More recently, flood-filling networks produce segmentations by training an end-to-end neural network that goes from EM images directly to label volumes~\cite{januszewski2016flood}. These flood-filling networks produce impressive accuracies but at a high computational cost.

\paragraph{Region-based methods.}
%Lately, many recent advancements in segmentation use artificial neural networks since machine-learned features outperform hand-designed ones~\cite{bogovic2013learned}.
%A significant amount of research focuses on the training and optimization of these deep neural networks~\cite{chatfield2014return,maas2013rectifier,nesterov1983method}.
Several pixel-based approaches generate probabilities that neighboring pixels belong to the same neuron.
Often a watershed algorithm will cluster pixels into small super-pixels~\cite{zlateski2015image}.
Many methods build on top of these strategies and train random-forest classifiers to produce region-based (i.e., super-pixel) segmentations~\cite{seymour2016rhoananet,nunez2014graph,10.1371/journal.pone.0125825,parag2017anisotropic,zlateski2015image}.

\paragraph{Segment-based methods.}
Some recent research builds on top of these region-based methods to correct errors in the segmentation~\cite{rolnick2017morphological,error_correction_using_CNN,haehn2017guided}.
However, to our knowledge, our method is the first to extract a 3D graph from pixel-based agglomerated segmentations for a true top-down reconstruction approach. This allows us to enforce domain-specific constraints using graph-based partitioning algorithms. Many segmentation and clustering algorithms use graph partitioning techniques~\cite{andres2012globally} or normalized cuts for traditional image segmentation~\cite{kappes2016higher,shi2000normalized,tatiraju2008image}.
Even though graph partitioning is an NP-Hard problem~\cite{demaine2006correlation} there are several useful multicut heuristics that provide good approximations with reasonable computational costs~\cite{horvnakova2017analysis}. We use the method of Keuper et al. to partition the extracted 3D graph into the final neuron reconstruction~\cite{keuper2015efficient}.


\section{Method}

Our method works on top of existing $EM$ agglomeration algorithms such as \textit{NeuroProof} or \textit{GALA} (CITE BOTH). First we use a 3D U-net to generate voxel affinities predicting the probability that two voxels belong to the same neuron. The \textit{zwatershed} algorithms takes these affinities and generates an oversegmentation of the neurons (CITE). \textit{NeuroProof} agglomerates these supervoxels by training a multi-level random forest classifier. We use a low threshold for neuroproof to generate an oversegmentation. 

NEED TO ADD BETTER OVERVIEW

\subsection{Preprocessing}

Our pipeline builds a level on top of current agglomeration techniques. Our goal is to identify and merge segments based on their overall shape. Figure (ADD FIGURE) shows two examples of nearby segments output by \textit{NeuroProof}. The right pair should merge and the left pair should not.  

We consider the skeletons of the segments. We use the teasFigure (ADD FIGURE) shows two examples of skeletons generated for two neuron segments. The skeletons are generated using the TEASER: Tree-structure Extraction Algorithm for Accurate and Robust Skeletons (CITE TEASER). The algorithm iteratively chooses locations that are the furthest distance $d$ from the boundary of the segment. Each locations becomes a ``joint" and a sphere of radius $d$ centered at the joint is masked out. The algorithm continually finds points and removes the corresponding sphere until every point with a distance greater than $d^*$ is removed. We use a $d^*$ of 50 voxels which was the standard parameter from the original paper (Is it?). 

We use the above skeletons to extract feature locations to consider merges. The algorithm is described in the pseudocode in (ADD CITE).

ADD IN PSEUDOCODE

\begin{itemize}
	
	
	
	\item Skeletonization
	\item Feature generation
\end{itemize}

\subsection{Error Detection}

\begin{figure*}[t]
	\centering
	\includegraphics[width=0.8\linewidth]{figures/architecture.png}
	\caption{The architecture for the neural networks follows the \textit{VGG} style of double convolutions followed by a max pooling operation. The number of filters doubles each layer leading to a fully connected layer and a sigmoid activation function.}
	\label{fig:architecture}
\end{figure*}

NEED TO CITE VGG, LeakyReLU, BatchNormalization

The above skeletonization strategies produces locations in 3D that require further consideration. From these locations we extract a region of $800nm \times 800nm \times 800nm$ and resample the segmentations to fit in a box of $18 \times 52 \times 52$ voxels. Consider a location corresponding to a potential merge between labels $l_1$ and $l_2$. The input to the neural network is a three binary channel cube of size $3 \times 18 \times 52 \times 52$. The first channel is $1$ if the voxel belongs to $l_1$. The second channel is 1 if the voxel belongs to $l_2$. The third channel is $1$ if the voxel belongs to either $l_1$ or $l_2$. 

We train a 3D convolutional neural network on the the generated examples. The network has four layers of double convolutions followed by a max pooling step. The filter sizes are $(3, 3, 3)$ and the max pooling is by a factor of $2$ for every level in the $x$ and $y$ dimensions. We downsample the $z$ dimension only in the final two layers. The number of channels is $16$ for the first layer and each subsequent layer has twice as many channels. The output of the convolutions is flattened and input into two dense layers. We apply batch normalization after every convolution, pooling, and dense layer except for the final layer. We use the \textit{LeakyReLU} activation function with $\alpha = 0.001$. We use the \textit{Adam} optimizer with an initial learning rate of $0.001$, $\beta_1 = 0.99$, $\beta_2 = 0.999$, and a weight decay of $5*10^{-8}$. The loss function is binary crossentropy. We use a batch size of 20.  

Since the datasets have a limited number of examples, we augment the examples by considering all combinations with rotations along the $xy$-plane and mirrors along the $yz$- and $xy$- planes. This produces 16 times more training data.

\begin{itemize}
	\item CNN Architecture
	\item Classifier Inputs
\end{itemize}

\subsection{Agglomeration}

The above neural network produces probabilities that two nearby segments belong to the same neuron. To use the graph-based optimization algorithms we need to take the oversegmentation and construct a graph. We generate one node for every segment. The skeletonization algorithm generates edges between nodes. 

\begin{itemize}
	\item Error Correction
\end{itemize}


% !TEX root =  paper.tex
\section{Experiments}

We evaluate our method by comparing it to a state-of-the-art pixel-based reconstruction approach using datasets from mouse and fly brains.

\subsection{Datasets}
\label{sec:dataset}

Our proposed method is designed for very large connectomics datasets. 
Popular challenge datasets such as CREMI and SNEMI3D are simply too small for any noticeable change. 
The following datasets contain 4 times more volume than the CREMI datasets. 

%\subsubsection{Kasthuri}
\noindent\textbf{Kasthuri}
The Kasthuri dataset consists of scanning electron microscope images of the neocortex of a mouse brain~\cite{kasthuri2015saturated}. 
This dataset is $5342 \times 3618 \times 338$ voxels in size. 
The resolution of the dataset is $\SI[product-units=single]{3 x 3 x 30}{\nano\meter}^3$ per voxel. 
We evaluate our methods using the left cylinder of this 3-cylinder dataset. 
We downsample the dataset in the $x$ and $y$ dimensions to give a final resolution of $\SI[product-units=single]{6 x 6 x 30}{\nano\meter}^3$ per voxel. 
We divide the dataset into two volumes (Vol. 1 and Vol. 2) along the $x$ dimension, where each volume is $\SI[product-units=single]{8.0 x 10.9 x 10.1}{\micro\meter}^3$ or $1335 \times 1809 \times 338$ voxels.

%\subsubsection{FlyEM}
\noindent\textbf{FlyEM}
The FlyEM dataset comes from the mushroom body of a 5-day old adult male \textit{Drosophila} fly imaged by a focused ion-beam milling scanning electron microscopy~\cite{takemura2017connectome}.
The mushroom body in this species is the primary site of associative learning. 
The original dataset contains a $\SI[product-units=single]{40 x 50 x 120}{\micro\meter}^3$ volume with a resolution of $\SI[product-units=single]{10 x 10 x 10}{\nano\meter}^3$ per voxel. 
We use two cubes (Vol. 1 and Vol. 2) of size $\SI[product-units=single]{10 x 10 x 10}{\micro\meter}^3$ or $999 \times 999 \times 999$ voxels.

%\subsubsection{Fib-25}
%\noindent\textbf{Fib-25}
%The Fib-25 dataset comes from the visual system of a \textit{Drosophila} fly~\cite{takemura2015synaptic}. This sample is %imaged at a resolution of $\SI[product-units=single]{8 x 8 x 8}{\nano\meter}^3$ and has $2588 \times 2108 \times 2876$ %voxels representing a volume of $\SI[product-units=single]{23.0 x 16.9 x 23.0}{\micro\meter}^3$. 
%There is currently a reconstruction challenge based on this sample.

\subsection{Method Configuration}
\noindent\textbf{Initial Segmentation}
%\label{sec:neuroproof}
%The segmentation of the Kasthuri dataset was computed by agglomerating 3D supervoxels produced by the watershed algorithm from 3D affinity predictions~\cite{zlateski2015image}. 
%A recent study by Funke et al.~\cite{funke2017deep} demonstrated superior performance of such methods over existing ones on anisotropic data. 
%We learn 3D affinities using MALIS loss with a U-net~\cite{ronneberger2015u,Turaga:2009}. 
%We apply the z-watershed algorithm with suitable parameters to compute a 3D oversegmentation of the volume. 
%The resulting 3D oversegmentation is then agglomerated using the technique of context-aware delayed agglomeration to generate the final segmentation~\cite{10.1371/journal.pone.0125825}.
For the Kasthuri dataset, we used the segmentation pipeline from Funke et al.~\cite{funke2017deep}. 
We first train a 3D affinity prediction U-Net model~\cite{ronneberger2015u} with MALIS loss~\cite{Turaga:2009} and apply the watershed and agglomeration method~\cite{funke2017deep} to obtain the final segmentation result.
For the FlyEM data, based on the authors' suggestion~\cite{takemura2017connectome}, we apply a context-aware delayed agglomeration algorithm~\cite{10.1371/journal.pone.0125825} that shows improved performance on this dataset over the pipeline used in the original publication. 
This segmentation framework learns voxel and supervoxel classifiers with an emphasis to minimize under-segmentation error. 
The algorithm first computes multi-channel 3-D predictions for membranes, cell interiors, and mitochondria, among other cell features. 
The membrane prediction channel is used to produce an over-segmented volume using 3D watershed, which is then agglomerated hierarchically up to a certain confidence threshold. 
We used exactly the same parameters as the publicly available code for this algorithm.
%For the Fib-25 challenge dataset, we use the segmentation from Funke et al. as our input~\cite{funke2017deep}.

\noindent\textbf{Graph Generation}
The two parameters for the graph pruning algorithm (Sec.~\ref{sec:skeletonization}) are $t_{low}$ and $t_{high}$. 
Ideally, our graph will have an edge for every over-segmented pair of labels with few edges between correctly segmented pairs. 
After considering various thresholds, we find that $t_{low} = \SI{210}{\nano\meter}$ and $t_{high} = \SI{300}{\nano\meter}$ produce expressive graphs with a scalable number of nodes and edges.
We explore these thresholds in greater detail in the supplemental material.
During implementation, we use nanometers instead of voxels for these thresholds to have uniform units across all datasets.
% Connectomics datasets often have lower sample resolutions in $z$ because of limitations during sample preparation. 
% Using nanometers allows us to have uniform units across all of these datasets and calculate the thresholds in voxels at runtime.
% Although the voxels vary between different datasets, the physical space remains constant. 


\noindent\textbf{Edge Weight Learning}
\label{sec:network-parameters}
We use half of the Kasthuri dataset for training and validation. 
We train on 80\% of the potential merge candidates for this volume.
We validate the CNN classifier on the remaining 20\% of the candidates. 
Since our input does not require the image data, we can train on the anisotropic Kasthuri data and test on the isotropic FlyEM data.

We consider networks with varying input sizes, optimizers, loss functions, filter sizes, learning rates, and activation functions. 
The supplemental material includes information on the experiments that determined these final parameters. 
We provide all parameters of the final network in the supplemental material. 
There are 2,313,969 learnable parameters in our final architecture. 
All the parameters are randomly initialized following the Xavier uniform distribution~\cite{glorot2010understanding}. 
Training concluded after 1000 epochs.

%\noindent\textbf{Training Augmentation.}
Since our input is an existing segmentation of the EM images, there are very few training examples compared to per-pixel classifiers that can train on a unique window for each voxel. 
The Kasthuri dataset represents a region of brain over $\SI[product-units=single]{800}{\micro\meter}^3$ in volume and only yields 640 positive merge examples and 4821 negative ones.
To avoid overfitting our deep networks, we apply the following augmentations on the training examples.
During a single batch, we randomly select ten positive and negative examples. 
With probability 0.5, the example is reflected across the $xy$-plane. 
We then rotate the example by a random angle between $0$ and $360$ degrees using nearest neighbor interpolation. 
The supplemental material contains experiments demonstrating the benefits of this augmentation strategy.
We have 20,000 such examples per epoch.

\noindent\textbf{Graph Partition}
We assign the edges in our graph a weight $w_e = \log{\frac{p_e}{1 - p_e}} + \log{\frac{1 - \beta}{\beta}}$ where $\beta$ is a tunable parameter. 
Higher values of $\beta$ encourage over-segmentation.
Figure~\ref{fig:variation-of-information} shows our results for variables $\beta \in [0.5, 0.9]$. 
For our multicut analysis we consider $\beta = 0.5$. 

\subsection{Error Metrics}
\label{sec:variation-of-information}
We evaluate the performance of the different methods using the split variation of information (VI)~\cite{meila2003comparing}.
Given a ground truth labeling $GT$ and our automatically reconstructed segmentation $SG$, over- and under-segmentation are quantified by the conditional entropies $H(GT | SG)$ and $H(SG | GT)$, respectively. 
Since we are measuring the entropy between two clusterings, lower VI scores are better.

We use precision and recall to evaluate the convolutional neural network and multicut outputs. 
Since our method attempts to only correct \textit{split errors}, we define a true positive as a pair of segments that are correctly merged together after our pipeline.

% !TEX root =  paper.tex

\begin{figure*}[t!]
	\centering
	\includegraphics[width=0.45\linewidth]{./figures/variation_of_information-microns-train-600.png}
	\includegraphics[width=0.45\linewidth]{./figures/variation_of_information-microns-test-600.png}
	\includegraphics[width=0.45\linewidth]{./figures/variation_of_information-FlyEM-train-600.png}
	\includegraphics[width=0.45\linewidth]{./figures/variation_of_information-FlyEM-test-600.png}
	\caption{VI scores of our method (red) compared to the baseline segmentation (green) and an oracle (blue) that optimally partitions the graph based on ground truth.}
	\label{fig:variation-of-information}
\end{figure*}

\section{Results}

\subsection{Error Metric}
\label{sec:variation-of-information}

We evaluate the performance of the different methods using the split version of variance of information (VI)~\cite{meila2003comparing}. 
Given a ground truth labeling $GT$ and our automatically reconstructed segmentation $SG$, over and undersegmentation are quantified by the conditional entropy $H(GT | SG)$ and $H(SG | GT)$, respectively. Since we are measuring the entropy between two clusterings, better VI scores are closer to the origin.

\subsection{Variation of Information Improvement}

In Fig.~\ref{fig:variation-of-information}, we show the VI results of NeuroProof on the Kasthuri and FlyEM data at varying thresholds of agglomeration (green). 
The green circle represents the variation of information for our input segmentation (a threshold of 0.3 for all datasets). 
Our results are in red.  
We show the comparison to an oracle (blue) that correctly partitions the graph from our algorithm based on ground truth. 

Our algorithm improves the accuracy of the input segmentation on every dataset, reducing the VI split score on average by \FIX{X\%} and only increasing the VI merge score by \FIX{X\%}. 
Scores closer to the origin are better for this metric, and in every instance we are below the green curve. 
We see significant improvements on the Kasthuri datasets (VI split reduction of \FIX{X\%} and \FIX{X\%} on the training and testing datasets respectively) and slightly more modest improvements on the FlyEM datasets (reduction of \FIX{X\%} and \FIX{X\%}).
However, our baseline, NeuroProof performs much better on the FlyEM datasets reducing the potential improvement. 
Isotropic datasets are easier to segment using state-of-the-art region-based methods than anisotropic ones~\cite{plaza2014annotating}.
Thus there is less room for improvement on these datasets. 

Fig.~\ref{fig:positive-results} shows successful merges on the Kasthuri Vol. 2 dataset. 
Several of these examples combine multiple consecutive segments that span the volume. 
In the third example we correct the oversegmentation of a dendrite.
Fig.~\ref{fig:negative-results} shows some failure cases. 
In two of these examples the algorithm correctly predicted several merges but made one error.
In the third example a merge error in the initial segmentation propagated to our output. 
We now analyze how each major component of our method contributes to this final result.

\begin{figure}[t]
	\centering
	\includegraphics[width=0.85\linewidth]{./figures/multicut-correct1.png}
	\includegraphics[width=0.85\linewidth]{./figures/multicut-correct2.png}
	\includegraphics[width=0.85\linewidth]{./figures/multicut-correct3.png}
	\includegraphics[width=0.85\linewidth]{./figures/multicut-correct4.png}
	\includegraphics[width=0.85\linewidth]{./figures/multicut-correct5.png}
	\caption{Correctly segmented neurons from our method.}
	\label{fig:positive-results}
\end{figure}

\begin{figure}[t]
	\centering
	\includegraphics[width=0.85\linewidth]{./figures/multicut-incorrect1.png}
	\includegraphics[width=0.85\linewidth]{./figures/multicut-incorrect2.png}
	\includegraphics[width=0.85\linewidth]{./figures/multicut-incorrect3.png}
	\includegraphics[width=0.85\linewidth]{./figures/multicut-incorrect4.png}
	\caption{Errors made by our method.}
	\label{fig:negative-results}
\end{figure}


\subsection{Graph Creation}

\begin{table}
	\centering
	\small
	\begin{tabular}{c c c} \hline
		\textbf{Dataset} & \textbf{Baseline} & \textbf{After Pruning} \\ \hline
		Kasthuri Vol. 1 & 763 / 21242 (3.47\%) & 753 / 3459 (17.88\%) \\
		Kasthuri Vol. 2 & 1010 / 26073 (3.73\%) & 904 / 4327 (17.28\%) \\
		FlyEM Vol. 1 & 269 / 14875 (1.78\%) & 262 / 946 (21.69\%) \\
		FlyEM Vol. 2 & 270 / 16808 (1.58\%) & 285 / 768 (27.07\%)\\ \hline
	\end{tabular}
	\caption{The results of our skeleton graph pruning heuristic compared to the baseline segmentation.}
	\label{table:skeletonization}
\end{table}

Table \ref{table:skeletonization} shows the results of pruning the skeleton graph using the heuristic discussed in Sec.~\ref{sec:skeletonization}. The baseline algorithm considers all adjacent regions for merging. Our method removes a significant portion of these candidates while maintaining a large number of the true merge locations. This edge pruning is essential for the graph partitioning algorithm, which has a computational complexity dependence on the number of edges. Our pruning heuristic removes at least $6\times$ the number of edges between correctly split segments on all datasets, achieving a maximum removal ratio of $20\times$.

Equally important is the number of split errors that remain after pruning. These are the locations that we want to merge to create a more accurate reconstruction. For every dataset, the number of true split errors remains constant before and after pruning.
However, since our heuristic does not enforce an adjacency constraint of two regions when constructing edges in the graph, the difference does not indicate the number of examples excluded by pruning. In fact, our method finds a number of examples that are non-adjacent. Figure \ref{fig:skeleton-results} shows two example segments with split errors, one that our algorithm missed (top) and one that it identified (bottom), despite the fact that the split segments are not adjacent.
Of the successful examples in Figure \ref{fig:positive-results}, the second and fourth groupings contain pairs of non-adjacent segments that merge with our algorithm.

\begin{figure}[h!]
	\centering
	\includegraphics[width=0.85\linewidth]{./figures/merge_candidate1.png}
	\includegraphics[width=0.85\linewidth]{./figures/merge_candidate2.png}
	\caption{Example merge candidates.}
	\label{fig:skeleton-results}
\end{figure}


\subsection{Classification Performance}

Figure \ref{fig:receiver-operating-characteristic} shows the receiver operating characteristic (ROC) curve for all of the datasets. 
We train our CNN using one of the Kasthuri volumes and test using the other three datasets.
Since our CNN only takes as input a region of the label volume we can train on an anisotropic dataset and infer on an isotropic one. 
This provides a major benefit given the time-intensive task of manually generating ground truth data at various resolutions.

As evidenced by the ROC curve, the test results on the Kasthuri data are better than the results for FlyEM.
We believe this is in part because of the differences in the datasets (i.e., isotropy and $xy$ resolution). 
To test this hypothesis, we also evaluate the performance of the FlyEM datasets when the network trains on FlyEM Vol. 1 and infers on FlyEM Vol. 2.\footnote{Since the FlyEM datasets have significantly fewer examples, we initialize the network with the weights from the Kasthuri training and have an initial learning rate of $10^{-4}$.}
The dotted lines in the figure represent these tests. 
There is a slight performance increase for FlyEM Vol. 2.
However, the FlyEM datasets have reasonable results when the network is trained by the Kasthuri data, and the results outside of this section follow from that setup.  

\begin{figure}
	\centering
	\includegraphics[width=0.95\linewidth]{./figures/receiver-operating-characteristic.jpg}
	\caption{The receiver operating characteristic for all four datasets.}
	\label{fig:receiver-operating-characteristic}
\end{figure}	
	
\subsection{Graph Based Strategies}

The graph-based optimization strategy increases our correction accuracy over using just the CNN. 
In particular, the precision increases on each dataset, although the recall decreases on all but one of the datasets. 
Since it is more difficult to correct merge errors than split errors, it is often desirable to sacrifice recall for precision. 
Table \ref{table:multicut} shows the changes in precision, recall, and accuracy for all four datasets compared to the CNN. 
Over the three testing datasets, applying a graph-based partitioning strategy reduced the number of merge errors by \FIX{X}, \FIX{Y}, and \FIX{Z}. 

\begin{table}[h]
	\centering
	\begin{tabular}{c c c c} \hline
		\textbf{Dataset} & $\Delta$ \textbf{Precision} & $\Delta$ \textbf{Recall} & $\Delta$ \textbf{Accuracy} \\ \hline
		Kasthuri Training & +3.60\% & -0.01\% & +0.60\% \\
		Kasthuri Testing & +7.59\% & -1.77\% & +1.38\% \\
		FlyEM Vol. 1 & +2.68\% & +0.76\% & +0.66\% \\
		FlyEM Vol. 2 & +2.22\% & -1.05\% & +0.29\% \\ \hline
	\end{tabular}
	\caption{Precision and recall for the training and three test datasets.}
	\label{table:multicut}
\end{table}


\section{Conclusions}

\begin{itemize}
	\item Impact
	\item Future work
\end{itemize}

	
\clearpage
	
\bibliographystyle{splncs}
\bibliography{egbib}

\end{document}
