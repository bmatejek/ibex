% !TEX root =  paper.tex
\section{Conclusions}

We present a novel method for improved neuronal reconstruction in connectomics that extends existing pixel-based reconstruction strategies using skeletonized 3D networks. 
We show significant accuracy improvements on datasets from two different species. 
The main benefits of our approach are that it enforces domain-specific constraints at the global graph level while incorporating pixel-based classification information.

In the future, these methods can be adjusted to apply additional domain constraints. 
We can augment the graph with more information from the image data, such as synaptic connections, cell morphology, and locations of mitochondria. 
This would allow us to match other biological constraints during graph partitioning. 
For example, we could then enforce the constraint that a given segment only has post- or pre-synaptic connections. 

An augmented graph would be helpful for splitting improperly merged segments by adding additional terms to the partitioning cost function. 
Using skeletons we can apply biological constraints on the topology for neurons that are improperly merged.
Finally, we believe that the benefits of top-down enhancements from graph optimization can extend beyond connectomics to other domains, such as medical image segmentation.